\section{Tests}
\begin{enumerate}[label=Test \alph*.,ref=Test \alph*,nosep]
	\item \label{testMassRate} Source/sink test: Mass rate
	\item \label{testScaledMassRateVol} Source/sink test: Scaled mass rate by volume
	\item \label{testScaledMassRatePerm} Source/sink test: Scaled mass rate by permeability
	\item \label{testVolRate} Source/sink test: Volume rate
	\item \label{testScaledVolRateVol} Source/sink test: Scaled volume rate by volume
	\item \label{testScaledVolRatePerm} Source/sink test: Scaled volume rate by permeability
	\item \label{test1DSatHyd} 1D fully saturated hydrostatic initial condition: Problem 2.2.6 from Kolditz et al. 2015
	\item \label{test1DSatDirBC} 1D fully saturated Dirichlet BC: Problem 2.2.7 from Kolditz et al. 2015
	\item \label{test1DSatNeumBC} 1D fully saturated Neumann BC: Problem 2.2.8 from Kolditz et al. 2015
	\item \label{test2DSatDirNeumBC} 2D fully saturated Dirichlet and Neumann BC:Problem 2.2.10 from Kolditz et al. 2015
	\item \label{test1DVarSatCelia} 1D variably saturated from Celia et al. 1990
	\item \label{test2DVarSat} 2D variably saturated: Infiltration in a large caisson: problem 10.13.3 from Feflow's manual
	\item \label{testEOSIFC67} EOS test for IFC67: compare density calculations between PFLOTRAN and python script from STOMP documentation
	\item \label{testEOSIF97} EOS test for IF97: compare density calculations between PFLOTRAN and online calculator
	\item \label{testRepMatCellIDStruc} Representation of material properties by cell ID on structured grid (with a 2x2x2 cube)
	\item \label{testRepMatCellIDStruc16} Representation of material properties by cell ID on structured grid using random correlated fields of porosity and permeability
	\item \label{testRepMatCellIDUnstruc} Representation of material properties by cell ID on unstructured grid (with a 2x2x2 cube)
	\item \label{testRepMatGridStruc} Representation of material properties by location using gridded datasets on structured grid
	\item \label{testRepMatGridUnstruc} Representation of material properties by location using gridded datasets on unstructured grid
	\item \label{testRepMatRegStruc} Representation of material properties by location using regions on structured grid
	\item \label{testRepMatRegUnstruc} Representation of material properties by location using regions on unstructured grid
	\item \label{testRepMatIJKStruc} Representation of material properties with IJK indices on structured grid
	\item \label{testInactive} Test capability of inactivating cells
	\item \label{testNonContMaterialIDS} Test non-contiguous material IDs
	\item \label{testStructIrregGrid} Test the ability to create structured grids with irregular spacing.
	\item \label{testRepICCellIDStruc} Representation of initial conditions by cell ID on structured grid (with a 2x2x2 cube)
\end{enumerate}

\begin{enumerate}[label=Test \greek*.,ref=Test \greek*,nosep]
\item \label{testRepICCellIDUnstruc} Representation of initial conditions by cell ID on unstructured grid (with a 2x2x2 cube)
\item \label{testRepICCellIDStruc16} Representation of initial conditions by cell ID on structured grid using random correlated fields of porosity, permeability, and initial pressure.
\item \label{testRepICRegStruc} Representation of initial conditions by location using regions on structured grid
\item \label{testRepICRegUnstruc} Representation of initial conditions by location using regions on unstructured grid
\item \label{testRepICIJKStruc} Representation of initial conditions with IJK indices on structured grid
\item \label{testRepICGridStruc} Representation of initial conditions by location using gridded datasets on structured grid
\item \label{testRepBCGridStruc} Representation of boundary conditions by location using gridded datasets on structured grid (shortcourse example  and comparison available at www.pflotran.org/qa)
\item \label{testTimeStepGrowth} Test time step variablity following the growth factor until it reaches the maximum time step size.
\item \label{testTimeStepReductCut} Test time step variablity following the reduction factor until it reaches the maximum number of consecutive cuts.
\item \label{testTimeStepReduct} Test time step variablity following the reduction factor until it reaches the minimum time step size.
\item \label{testUserMailingList} \url{https://groups.google.com/g/pflotran-users}
\item \label{testDevMailingList} \url{https://groups.google.com/g/pflotran-dev}
\end{enumerate}



	
