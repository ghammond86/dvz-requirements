\section{Introduction}
This specification documents features required by PFLOTRAN to simulate physical and chemical processes within the subsurface environment. These required features are divided into functional and non-functional requirements. Functional requirements consist of physical or chemical processes, numerical methods, user interfaces, etc.  Non-functional requirements include runtime performance metrics (e.g. scalability,  performance), software maintenance and code availability. The specification maps these features to tests designed to verify their accuracy and robustness.

\section{Functional Requirements}
PFLOTRAN’s functional requirements are divided into the following categories: constitutive relations, physics and chemistry, numerical methods, and user interaction.
\subsection{Constitutive Relations}

Constitutive relations employ mathematical equations to approximate the observed physical response of a material or fluid under conditions of interest. For instance, an equation of state calculates a fluid density as a function of input parameters pressure and temperature.

\subsubsection{Equations of State}
Equations of state calculate fluid density, enthalpy, and viscosity as a function of temperature and pressure. The following equations of state shall be implemented in the code:
\begin{enumerate}[label=CR \arabic*.,ref=CR \arabic*,nosep]
	\item IFC67: \label{ifc67} PFLOTRAN shall implement the IFC67 (International Formulation Committee, 1967) equations of state for calculating the density, enthalpy, and viscosity of water as a function of pressure and temperature.
	\item IF97: \label{if97} PFLOTRAN shall implement the IF97 from the International Association for the Properties of Water and Steam (Wagner et al, 2000) equations of state for calculating the density and enthalpy of water and steam as a function of pressure and temperature.
\end{enumerate}

\subsubsection{Capillary Pressure/Saturation Functions}
In variably saturated or multiphase flow in porous media it is essential to establish a relation between capillary pressure and saturation. This relation is set forth by several empirical models, and two of the more common are implemented in PFLOTRAN:
\begin{enumerate}[label=CR \arabic*.,ref=CR \arabic*,nosep, resume]
	\item van Genuchten: \label{vanGen} PFLOTRAN shall implement the van Genuchten function (van Genuchten, 1980) to calculate the saturation on a cell given its capillary pressure value.
	\item Brooks-Corey: \label{brCorey} The simulator may use the Brooks-Corey function (Brooks and Corey,  1964) to calculate the saturation on a cell given its capillary pressure value.
\end{enumerate}

\subsubsection{Relative Permeability Functions}

Relative permeability functions establish a relationship between liquid relative permeability [-] (which varies from 0 to 1) and saturation. The simulator shall implement the following relative permeability functions: 

\begin{enumerate}[label=CR \arabic*.,ref=CR \arabic*,nosep, resume]
	\item Mualem: \label{mualem} The simulator calculates relative permeability as a function of saturation using either the Mualem-van Genuchten or the Mualem-Brooks-Corey functions (Chen et al. 1990). 
	\item Burdine: \label{burdine} The simulator calculates relative permeability as a function of saturation using either the Burdine-van Genuchten or the Burdine-Brooks-Corey functions (Chen et al. 1990). 
\end{enumerate}

\subsubsection{Soil Compressibility Functions}
The storage of fluids in porous media change according to compaction or expansion of fluids and porous media. While compaction (or expansion) of fluids is governed by the fluid compressibility, the changes in porous media volume is set forth by soil compressibility functions. These functions relate changes in porosity with changes in pressure. PFLOTRAN shall implement two soil compressibility functions:
\begin{enumerate}[label=CR \arabic*.,ref=CR \arabic*,nosep, resume]
	\item Leijnse: \label{leijnse} Leijnse function (Leijnse 1992) is the default soil compressibility function considered by the simulator. It calculates the current porosity as a function of pressure, given soil matrix compressibility coefficient, a reference pressure, and reference porosity.
	\item Exponential: \label{exponential} An exponential soil compressibility function establishes an exponential relationship between changes in pressure and porosity, given a soil matrix compressibility coefficient, a reference pressure, and reference porosity. (Low - easy)
\end{enumerate}
\subsubsection{Constitutive Relation Coupling by Region}
\begin{enumerate}[label=CR \arabic*.,ref=CR \arabic*,nosep, resume]
	\item PFLOTRAN shall allow for capillary pressure/saturation functions, relative permeability functions and soil compressibility functions to be specified by region. \label{CRbyRegion} (HIGH - easy)
\end{enumerate}

\subsection{Physics and Chemistry}
PFLOTRAN employs mathematical representations of physical and chemical process models to simulate phenomenon in the subsurface.  These process models include:
\subsubsection{Single phase variably saturated flow}
\begin{enumerate}[label=PC \arabic*.,ref=PC \arabic*,nosep]
	\item The governing mass balance equation that is used to model single phase variably saturated flow in PFLOTRAN is based on Richards’ equation (see Richards 1931 and Freeze and Cherry 1979). \label{sngVarSatFlow}
\end{enumerate}
\todo{Should we add single phase fully saturated flow?}
\subsubsection{Multicomponent Solute Transport}
\subsubsection{Biogeochemical Reaction}

\subsection{Forcing Requirements (Boundary conditions, initial conditions and source/sinks)}
PFLOTRAN shall allow the specification of the following flow boundary conditions:


\subsubsection{Liquid Pressure}
\begin{enumerate}[label=FR \arabic*.,ref=FR \arabic*,nosep]
	\item Dirichlet: \label{dirichlet} Pressure Dirichlet boundary conditions specify a particular pressure at the boundaries. 
	\item Pressure Hydrostatic: \label{presHyd} PFLOTRAN allows the specification of a pressure hydrostatic boundary or initial condition, where the pressure ($p$) is a function of the fluid density ($\rho$), the gravity ($g$), and the distance from a datum ($h$), that is $p= \rho gh$, where $\rho$ may be a function of pressure and temperature. The hydrostatic pressure profile may also be assigned based on a gradient in the horizontal direction.
\end{enumerate}

\subsubsection{Liquid Flux}
\begin{enumerate}[label=FR \arabic*.,ref=FR \arabic*,nosep, resume]
	\item Neumann: \label{neumann} Flux Neumann boundary conditions specify a Darcy flux [m/s] across a boundary.

\end{enumerate}

\subsubsection{Liquid Source/Sink}
\begin{enumerate}[label=FR \arabic*.,ref=FR \arabic*,nosep, resume]
	\item Mass rate: \label{massRate} The mass rate shall be equally distributed across cells in the region where the source/sink term is applied.
	\item: Scaled mass rate by cell volume: \label{scaledMassRateVol} The mass rate shall be distributed according to the ratio ($r$) of the cell volume ($V_c$) by the total volume of the region ($V_t$) where the source/sink term is applied to. The mass rate is scaled by the ratio $r= V_c⁄V_t$.
	\item Scaled mass rate by cell permeability: \label{scaledMassRatePerm} The mass rate shall be scaled according to the ratio ($r$) of cell volume ($V_c$) multiplied by the cell’s intrinsic permeability ($\kappa_c$) and the sum of every cell volume in the region scaled by its permeability, that is $r = V_c \cdot \kappa_c \, / \sum_{n=1}^{n_r} V_n \cdot \kappa_n$, where $n_r$ is the total number of cells in the region where the source/sink is applied.
	\item Volumetric rate: \label{volRate} The volumetric rate shall be equally distributed across cells in the region where the source/sink term is applied.
	\item Scaled volumetric rate by cell volume: \label{scaledVolRateVol} The volume rate shall be distributed according to the ratio ($r$) of the cell volume ($V_c$) by the total volume of the region ($V_t$) where the source/sink term is applied to, that is $r=  V_c ⁄ V_t$.
	\item Scaled volumetric rate by cell permeability: \label{scaledVolRatePerm} The volumetric rate shall be scaled according to the ratio ($r$) of cell volume ($V_c$) multiplied by the cell’s intrinsic permeability ($\kappa_c$) and the sum of every cell volume in the region scaled by its permeability, that is $r = V_c \cdot \kappa_c \, / \sum_{n=1}^{n_r} V_n \cdot \kappa_n$, where $n_r$ is the total number of cells in the region where the source/sink is applied.
\end{enumerate}

\begin{enumerate}[label=FR \arabic*.,ref=FR \arabic*,nosep, resume]
	\item The values applied to flow boundary conditions and source/sink terms may vary in time. 
\label{FRvaryTime}
	\item If no boundary condition is specified, a no-flux condition is assumed. \label{FRnoFlow}
\end{enumerate}

\subsection{Numerical Methods}

\subsubsection{Time Stepping}
\begin{enumerate}[label=NM \arabic*.,ref=NM \arabic*,nosep]
	\item Variable Time Stepping: \label{NMvarTS} PFLOTRAN shall have the ability to vary the time stepping. The time stepping will depend on the initial time step size, the minimal and maximum time step size, and the maximum growth and reduction factor. The maximum time step size may change during the simulation time. Time steps are increased or reduced according to growth and reduction factor as a function of the number of iterations needed for convergence. HIGH - medium
	\item Time Step Restriction by CFL: \label{TSbyCFL} PFLOTRAN shall have the ability to restrict time steps as a function of the maximum flow velocity and grid discretization such that CFL (Courant–Friedrichs–Lewy) number is not exceeded. HIGH - easy
	\item Variable Time Step Size by Process Model: \label{TSbyModel} PFLOTRAN shall allow different process models, such as flow and transport, to comply with different time step settings. MEDIUM - easy
\end{enumerate}

\subsubsection{Nonlinear solvers}
\begin{enumerate}[label=NM \arabic*.,ref=NM \arabic*,nosep, resume]
	\item PFLOTRAN shall implement a Newton-Raphson strategy to solve the set of nonlinear governing equations and iteratively drive the norm of the residual vector to below a desired convergence tolerance. \label{nonlinearSolver} LOW - easy
	\item The software shall report convergence failure and cut the timesetp size if the maximum number of Newton iterations is reached. \label{nonlinearReport} LOW-easy
\end{enumerate}

Convergence may be verified by five different convergence criteria, named as follows:
\begin{enumerate}[label=NM \arabic*.,ref=NM \arabic*,nosep, resume]
	\item ATOL: \label{nonlinearATOL} Convergence is met when the 2-norm of residual is less than ATOL.
	\item DTOL:\label{nonlinearDTOL} DTOL establishes divergence when the 2-norm of the residual is greater than DTOL multiplied by the 2-norm of the initial residual (relative to the first Newton iteration).
	\item ITOL\_UPDATE:\label{nonlinearITOL} Convergence is met when the infinity norm update (difference between the current and previous solution) is less than ITOL\_UPDATE.
	\item RTOL:\label{nonlinearRTOL} Convergence is met when the 2-norm of residual is less than RTOL multiplied by the 2-norm of the residual from the first Newton iteration.
	\item STOL:\label{nonlinearSTOL} Convergence is met when the 2-norm of the update (difference between the current and the previous iteration solution) is less than STOL multiplied by the 2-norm of the previous iteration solution.
\end{enumerate}
MEDIUM - easy

\subsubsection{Linear solvers}
The set of nonlinear governing equations is solved using Newton-Raphson’s method, which uses a sequence of linearized problems to find the solution. To solve this system of linear equations, PFLOTRAN allows the use of two solvers: direct and iterative solvers. The direct solver uses an LU decomposition and iterative solver options are: Bi-CGStab (default), and GMRES. PFLOTRAN allows the specification of the maximum number iterations. 
For iterative solvers, three types of convergence criteria shall be implemented:
\begin{enumerate}[label=NM \arabic*.,ref=NM \arabic*,nosep, resume]
	\item ATOL: \label{linearATOL} Convergence is met when the 2-norm of residual is less than ATOL.
	\item DTOL:\label{linearDTOL} DTOL establishes divergence when the 2-norm of the residual is greater than DTOL multiplied by the 2-norm of the initial residual (relative to the first linear iteration).
	\item RTOL:\label{linearRTOL} Convergence is met when the 2-norm of residual is less than RTOL multiplied by the 2-norm of the residual from the first linear iteration.
\end{enumerate}
MEDIUM - easy

\subsubsection{Finite volume implementation}
\begin{enumerate}[label=NM \arabic*.,ref=NM \arabic*,nosep, resume]
	\item The governing equations are discretized in PFLOTRAN using a cell-centered finite volume approach, and a two-point flux approximation (TPFA) scheme is employed to discretize the mass flux between two grid cells. \label{FVmethod} 
\end{enumerate}

\subsubsection{Gridding}
PFLOTRAN shall accept the following type of grids:
\begin{enumerate}[label=NM \arabic*.,ref=NM \arabic*,nosep, resume]
	\item Structured grids:  \label{structGrid} PFLOTRAN has the capacity of creating structured cartesian grids of hexahedron cells with varying grid spacing.
	\item Implicit unstructured grids: \label{impStructGrid} An implicit unstructured grid allows the domain to be discretized with the following types of cells: tetrahedron (4 vertices), pyramid (5 vertices), wedge (6 vertices), and hexahedron (8 vertices). The implicit grid cells are defined by a list of vertices numbers and the vertices are defined by their coordinates.
	\item Explicitly unstructured grids: \label{expStructGrid} An explicit unstructured grid allows the domain to be discretized with all types of cells, including Voronoi cells. The grid is described by a list of cells and connectivity. Cells are defined by an id, the cell-center coordinates, and the cell volume. The connectivity between two cells is composed by the id for each cell, the area that connects the cells and the face-center coordinates.
\end{enumerate}
\todo{We have a test that checks if structured irregular grids with a shifted origin is well built. Do we need a requirement for that?}
\subsection{Representation of Material Properties}
PFLOTRAN easily handles highly heterogenous data. PFLOTRAN has the capability of assigining material properties such as permeability and porosity on a cell by cell basis.

\begin{enumerate}[label=RMP \arabic*.,ref=RMP \arabic*,nosep]
	\item PFLOTRAN has the capability of inactiving cells in a determined region.
	\item Material properties are assembled within groups and an integer  material ID is assigned to each group for identification. Material IDs can be non-contiguous integers numbers.
\end{enumerate}

PFLOTRAN shall assign material properties and constitutive relationship on a heterogeneous domain using the following strategies:
\begin{enumerate}[label=RMP \arabic*.,ref=RMP \arabic*,nosep, resume]
	\item By region: \label{repMatPropRegions} PFLOTRAN allows the definitions of different regions in the domain. Each region can be linked with a material property.
	\item  By location (x,y,z coodinates): \label{repMatPropLocation} PFLOTRAN allows the specification of different material properties based on the cells coordinates.
	\item  By cell IDs: \label{repMatPropCellID} PFLOTRAN allows the specification of different material properties based on the cells ID numbers.
\end{enumerate}

\subsection{Representation of Initial Conditions}
PFLOTRAN has the capability of assigining different initial conditions on a cell by cell basis.

PFLOTRAN shall assign heterogeneous initial conditions across the domain using the following strategies:
\begin{enumerate}[label=RIC \arabic*.,ref=RIC \arabic*,nosep]
	\item By region: \label{repICRegions} Several initial conditions are attributed based on the definition of different domain regions.
	\item By location (x,y,z coodinates): \label{repICLocation} Varying initial conditions are specified based on the location of the cells (and their coordinates).
	\item By cell IDs: \label{repICCellID} Initial conditions are set up based on cells IDs.
\end{enumerate}
HIGH - easy
\subsection{Representation of Boundary Conditions and Source/Sink terms}
PFLOTRAN has the capability of assigining different boundary conditions and source-sink terms on a cell by cell basis.

PFLOTRAN shall allow the specification of various boundary conditions and source-sink terms using the following strategies:
\begin{enumerate}[label=RBC \arabic*.,ref=RBC \arabic*,nosep]
	\item By region: \label{repBCRegions} Boundary conditions may be linked with different domain regions.
	\item  By location (x,y,z coodinates): \label{repBCLocation} Boundary conditions may vary along the boundaries based on the cells coordinates.
\end{enumerate}

\subsection{User Interaction}
\subsubsection{Input format}
\begin{enumerate}[label=UI \arabic*.,ref=UI \arabic*,nosep]
	\item PFLOTRAN reads an ASCII file as input. The input file is divided into blocks and sub-blocks. The block that specifies the type of simulation to run (e.g.: subsurface) and the process model to use (e.g.: subsurface flow or subsurface transport) is called the simulation block. The simulation block is required in every input file. The remaining blocks define numerical methods, solver options, domain discretization, material properties, constitutive relations, time step options, output options, initial and boundary conditions and regions within the domain. \label{inputFile} LOW - easy
	\item PFLOTRAN allows the definition of regions within the domain, which can be cuboids, rectangles or points. These regions may be linked with different material properties, initial and boundary conditions, allowing those to be linked on a cell by cell basis. \label{inputRegions} HIGH - easy
	\item Except for temperature (C), default units and values shall be assumed when no specification is inserted in the input file. \label{inputUnitsValues} LOW - hard
	\item If any required keyword, block, sub-block, or property is missing, the simulation will throw an error message informing the user about their mistake. \label{inputErrors} LOW - hard
\end{enumerate}

\subsubsection{Output format}
\begin{enumerate}[label=UI \arabic*.,ref=UI \arabic*,nosep, resume]
	\item Output files may be generated for specific moments in time or for periodic times or time steps. \label{outputSpecs} HIGH - easy
	\item Output results are printed in screen for specified periodic time steps. Screen output may show the 2-norm of the residual, solution, update, and the infinity norm of the residual and update for every Newton iteration. It also shows the number of linear and non-linear iterations needed to reach convergence. \label{outputPrintScreen} LOW - easy
	\item The screen output is saved in an ASCII file, which also contains a summary of all parameters and problem setup inputs. \label{outputFile} HIGH - easy
\end{enumerate}

PFLOTRAN outputs the following results: 

\begin{enumerate}[label=UI \arabic*.,ref=UI \arabic*,nosep, resume]
	\item Snapshot file: \label{outSnapshot}A snapshot file outputs the value of specified variables over the entire domain at a specific time. Several file formats shall serve as snapshot files: Tecplot block, Tecplot point, VTK, and HDF5 (default). HIGH - easy
	\item Observation file: \label{outObservation}An observation file outputs the values of specified variables at a determined point over prescribed times. The observation file format in ASCII columns.HIGH - easy
	\item Mass balance file: \label{outMassBalance}A mass balance output file returns the global mass balance and the fluxes at all boundaries for water at specified times using an ASCII file.HIGH - easy
\end{enumerate}

\section{Non-Functional Requirements}
\subsection{Runtime Performance}
\begin{enumerate}[label=NFR \arabic*.,ref=NFR \arabic*,nosep]
	\item The code shall report the total run time at the end of simulation and record the number of processes employed to run the problem. LOW - easy
\end{enumerate}

\subsection{Software Maintenance}
\begin{enumerate}[label=NFR \arabic*.,ref=NFR \arabic*,nosep, resume]
	\item PFLOTRAN shall use a distributed version control to promote a collaborative environment for software development. The collaborative environment is facilitated by the possibility to have remote repositories, which encourages developers to work using several workflow configurations. Distributed version control tracks changes in the code and allows developers to have the full history record.\label{versionControl}
	\item All code shall meet the standards provided in the Developer’s guide. \label{codeStandard}
	\item PFLOTRAN developers’ guide shall provide instructions for reporting bugs.\label{reportBugs}
	\item Automated testing shall be implemented for any new capability added.\label{autTesting}
	\item Any user may contribute to PFLOTRAN code. Changes to the code must undergo peer review before being accepted and must pass all unit and regression tests.\label{contribution}
\end{enumerate}

\subsection{User support}
\begin{enumerate}[label=NFR \arabic*.,ref=NFR \arabic*,nosep, resume]
	\item PFLOTRAN shall provide a channel to support to users. This channel may be used for reporting bugs, asking and answering questions, as well as creating a connected community.\label{userSupport}
\end{enumerate}

\subsection{Code Availability}
\begin{enumerate}[label=NFR \arabic*.,ref=NFR \arabic*,nosep, resume]
	\item PFLOTRAN is a free software and available in an open access repository.\label{codeAvail}
\end{enumerate}




















